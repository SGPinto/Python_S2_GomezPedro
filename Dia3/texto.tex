
\documentclass{jornada} % Classe do documento

\usepackage{enumitem,lipsum}
\usepackage{listings}
\usepackage{color}

\definecolor{dkgreen}{rgb}{0,0.6,0}
\definecolor{gray}{rgb}{0.5,0.5,0.5}
\definecolor{mauve}{rgb}{0.58,0,0.82}

\lstset{frame=tb,
  language=Python,
  aboveskip=3mm,
  belowskip=3mm,
  showstringspaces=false,
  columns=flexible,
  basicstyle={\small\ttfamily},
  numbers=none,
  numberstyle=\tiny\color{gray},
  keywordstyle=\color{blue},
  commentstyle=\color{dkgreen},
  stringstyle=\color{mauve},
  breaklines=true,
  breakatwhitespace=true,
  tabsize=3
}

\titulo{Explicación de Comprensión de Listas}
\author[1*]{Pedro Felipe Gómez Bonilla} % Usar * para identificar o autor correspondente
\affil[1]{Campuslands}
\email{pedrogomez.campuslands@gmail.com} % colocar o e-amil do autor correspondente
\GrupoTematico{Grupo S2 - Campuslands} % Inserir o grupo temático

%------------------------------------------------------
% INÍCIO DO DOCUMENTO                                           %------------------------------------------------------        
\begin{document}
\pretext % não apague esta linha

% INÍCIO DA PARTE TEXTUAL                                           

\section{Introducción}

Cuando hablamos de comprensión de listas hacemos referencia a crear una lista de elementos con solamente una línea de código. Esto se puede evidenciar de la siguiente manera:
\begin{lstlisting}
// main.py
lista=[]
lista = [i**2 for i in range(5)]
\end{lstlisting}
\begin{subequations}

\end{subequations}

Aún asi , el ejercicio nos solicita los siguientes requerimientos: Generar diferentes tripletas conformadas como \([i,j,k]\) , las cuales van a asumir el valor de \(x\) , \(y\), y \(z\) ingresados por el usuario respectivamente. Dichas asignaciones deben cumplir con las siguientes condicionales: \(x\) , \(y\), y \(z\) deben ser menores o iguales a \(i\),\(j\) y \(k\) respectivamente, al igual que deben ser menor o igual a los mismos(\(i\leq x\),\(j\leq y\),\(k\leq z\)). También la sumatoria de \(i+j+k\) debe ser diferente a un numero dado por el usuario, el cual llamaremos \(n\).

Vamos a empezar a hacer un ejemplo donde \(x=1\), \(y=1\),\(z=2\) y \(n=3\). En otras palabras, generaremos tipletas del formato \([i,j,k]\), donde iteraremos entre \(x\), \(y\) y \(z\) , en las cuales su sumatoria sea diferente a \(n\), el cual tiene un valor de \(3\). Dichas tripletas la almacenaremos en una lista vacia \([ ]\), la cual imprimiremos al final de ejercicio.

Para ello vamos a empezar creando una tripleta con los valores de \([i,j,k]\) vacios, en otras palabras, debería quedar así: \([0,0,0]\). Luego empezaremos con el primer término \(x\), el cual utilizaremos en la posición \(k\), cumpliendo la condición anteriormente mencionada. En este caso \(k\) será igual al valor de 1 y \(z\)

\end{document}